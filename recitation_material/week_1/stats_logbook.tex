\documentclass[12pt]{article}
\usepackage{float}
\usepackage{amssymb}
\usepackage{graphicx}
\usepackage{amsmath}
\usepackage[margin=1in]{geometry}
\usepackage{indentfirst}
\usepackage{array}
\usepackage{titling}
\usepackage{hyperref}
\usepackage{enumitem}
\usepackage{setspace}
\DeclareUnicodeCharacter{202F}{\,}

\title{Useful Statistical Definitions}
\author{}
\date{}

\begin{document}
\maketitle

\section*{Purpose of Document}

\noindent This \LaTeX document provides an initial example of how to keep a statistical definitions logbook. Given that you will consistently be encountering unfamiliar statistical and mathematical concepts as you read methodology papers for your own research--- regardless of your existing knowledge of both disciplines--- it will be quite useful to have an ongoing list of statistical defintions that you can refer to when reading method papers. The purpose of such a document isn't to define and list every statistical term and probability distribution you encounter, but rather is most efficiently used for terms you would need to look up if you were to read them in a methods paper.

\section*{Definitions}
\subsection*{Probability}

\begin{quote}
\textbf{Estimator}--- A rule (such as a formula) which uses sample data to estimate the value of an unknown population parameter

\textbf{Parameter}--- A ``true value", it's a quantity of interest that summarizes some aspect of a population, like the mean, mode, or standard deviation. 

\textbf{Probability Distribution}--- A mathematical function/description of a probability of events for a random variable. It can be discrete, like a Poisson or Bernoulli distribution, or continuous, like a normal distribution or Gamma distribution.
\end{quote}

\subsection*{Regressions}

\begin{quote}
\textbf{Standard Error}--- The standard deviation of the sampling distribution of a statistic. Usually encountered as the standard deviation of a regression coefficient, used to calculate confidence intervals.
\end{quote}

\subsection*{Research Design}

\begin{quote}
\textbf{Representation sample}--- A sample which is a subset of a larger population that matches, roughly, the characteristics of the larger population
    
\textbf{External Validity}--- The ability to draw conclusions from a specific study or context to broader contexts which are not considered within the study or experiment itself.
\end{quote}

\end{document}