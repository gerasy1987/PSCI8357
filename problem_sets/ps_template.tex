% Options for packages loaded elsewhere
% Options for packages loaded elsewhere
\PassOptionsToPackage{unicode}{hyperref}
\PassOptionsToPackage{hyphens}{url}
\PassOptionsToPackage{dvipsnames,svgnames,x11names}{xcolor}
%
\documentclass[
  11pt,
  letterpaper,
  DIV=11,
  numbers=noendperiod]{scrartcl}
\usepackage{xcolor}
\usepackage[margin=1in]{geometry}
\usepackage{amsmath,amssymb}
\setcounter{secnumdepth}{-\maxdimen} % remove section numbering
\usepackage{iftex}
\ifPDFTeX
  \usepackage[T1]{fontenc}
  \usepackage[utf8]{inputenc}
  \usepackage{textcomp} % provide euro and other symbols
\else % if luatex or xetex
  \usepackage{unicode-math} % this also loads fontspec
  \defaultfontfeatures{Scale=MatchLowercase}
  \defaultfontfeatures[\rmfamily]{Ligatures=TeX,Scale=1}
\fi
\usepackage{lmodern}
\ifPDFTeX\else
  % xetex/luatex font selection
\fi
% Use upquote if available, for straight quotes in verbatim environments
\IfFileExists{upquote.sty}{\usepackage{upquote}}{}
\IfFileExists{microtype.sty}{% use microtype if available
  \usepackage[]{microtype}
  \UseMicrotypeSet[protrusion]{basicmath} % disable protrusion for tt fonts
}{}
\makeatletter
\@ifundefined{KOMAClassName}{% if non-KOMA class
  \IfFileExists{parskip.sty}{%
    \usepackage{parskip}
  }{% else
    \setlength{\parindent}{0pt}
    \setlength{\parskip}{6pt plus 2pt minus 1pt}}
}{% if KOMA class
  \KOMAoptions{parskip=half}}
\makeatother
% Make \paragraph and \subparagraph free-standing
\makeatletter
\ifx\paragraph\undefined\else
  \let\oldparagraph\paragraph
  \renewcommand{\paragraph}{
    \@ifstar
      \xxxParagraphStar
      \xxxParagraphNoStar
  }
  \newcommand{\xxxParagraphStar}[1]{\oldparagraph*{#1}\mbox{}}
  \newcommand{\xxxParagraphNoStar}[1]{\oldparagraph{#1}\mbox{}}
\fi
\ifx\subparagraph\undefined\else
  \let\oldsubparagraph\subparagraph
  \renewcommand{\subparagraph}{
    \@ifstar
      \xxxSubParagraphStar
      \xxxSubParagraphNoStar
  }
  \newcommand{\xxxSubParagraphStar}[1]{\oldsubparagraph*{#1}\mbox{}}
  \newcommand{\xxxSubParagraphNoStar}[1]{\oldsubparagraph{#1}\mbox{}}
\fi
\makeatother

\usepackage{color}
\usepackage{fancyvrb}
\newcommand{\VerbBar}{|}
\newcommand{\VERB}{\Verb[commandchars=\\\{\}]}
\DefineVerbatimEnvironment{Highlighting}{Verbatim}{commandchars=\\\{\}}
% Add ',fontsize=\small' for more characters per line
\usepackage{framed}
\definecolor{shadecolor}{RGB}{241,243,245}
\newenvironment{Shaded}{\begin{snugshade}}{\end{snugshade}}
\newcommand{\AlertTok}[1]{\textcolor[rgb]{0.68,0.00,0.00}{#1}}
\newcommand{\AnnotationTok}[1]{\textcolor[rgb]{0.37,0.37,0.37}{#1}}
\newcommand{\AttributeTok}[1]{\textcolor[rgb]{0.40,0.45,0.13}{#1}}
\newcommand{\BaseNTok}[1]{\textcolor[rgb]{0.68,0.00,0.00}{#1}}
\newcommand{\BuiltInTok}[1]{\textcolor[rgb]{0.00,0.23,0.31}{#1}}
\newcommand{\CharTok}[1]{\textcolor[rgb]{0.13,0.47,0.30}{#1}}
\newcommand{\CommentTok}[1]{\textcolor[rgb]{0.37,0.37,0.37}{#1}}
\newcommand{\CommentVarTok}[1]{\textcolor[rgb]{0.37,0.37,0.37}{\textit{#1}}}
\newcommand{\ConstantTok}[1]{\textcolor[rgb]{0.56,0.35,0.01}{#1}}
\newcommand{\ControlFlowTok}[1]{\textcolor[rgb]{0.00,0.23,0.31}{\textbf{#1}}}
\newcommand{\DataTypeTok}[1]{\textcolor[rgb]{0.68,0.00,0.00}{#1}}
\newcommand{\DecValTok}[1]{\textcolor[rgb]{0.68,0.00,0.00}{#1}}
\newcommand{\DocumentationTok}[1]{\textcolor[rgb]{0.37,0.37,0.37}{\textit{#1}}}
\newcommand{\ErrorTok}[1]{\textcolor[rgb]{0.68,0.00,0.00}{#1}}
\newcommand{\ExtensionTok}[1]{\textcolor[rgb]{0.00,0.23,0.31}{#1}}
\newcommand{\FloatTok}[1]{\textcolor[rgb]{0.68,0.00,0.00}{#1}}
\newcommand{\FunctionTok}[1]{\textcolor[rgb]{0.28,0.35,0.67}{#1}}
\newcommand{\ImportTok}[1]{\textcolor[rgb]{0.00,0.46,0.62}{#1}}
\newcommand{\InformationTok}[1]{\textcolor[rgb]{0.37,0.37,0.37}{#1}}
\newcommand{\KeywordTok}[1]{\textcolor[rgb]{0.00,0.23,0.31}{\textbf{#1}}}
\newcommand{\NormalTok}[1]{\textcolor[rgb]{0.00,0.23,0.31}{#1}}
\newcommand{\OperatorTok}[1]{\textcolor[rgb]{0.37,0.37,0.37}{#1}}
\newcommand{\OtherTok}[1]{\textcolor[rgb]{0.00,0.23,0.31}{#1}}
\newcommand{\PreprocessorTok}[1]{\textcolor[rgb]{0.68,0.00,0.00}{#1}}
\newcommand{\RegionMarkerTok}[1]{\textcolor[rgb]{0.00,0.23,0.31}{#1}}
\newcommand{\SpecialCharTok}[1]{\textcolor[rgb]{0.37,0.37,0.37}{#1}}
\newcommand{\SpecialStringTok}[1]{\textcolor[rgb]{0.13,0.47,0.30}{#1}}
\newcommand{\StringTok}[1]{\textcolor[rgb]{0.13,0.47,0.30}{#1}}
\newcommand{\VariableTok}[1]{\textcolor[rgb]{0.07,0.07,0.07}{#1}}
\newcommand{\VerbatimStringTok}[1]{\textcolor[rgb]{0.13,0.47,0.30}{#1}}
\newcommand{\WarningTok}[1]{\textcolor[rgb]{0.37,0.37,0.37}{\textit{#1}}}

\usepackage{longtable,booktabs,array}
\usepackage{calc} % for calculating minipage widths
% Correct order of tables after \paragraph or \subparagraph
\usepackage{etoolbox}
\makeatletter
\patchcmd\longtable{\par}{\if@noskipsec\mbox{}\fi\par}{}{}
\makeatother
% Allow footnotes in longtable head/foot
\IfFileExists{footnotehyper.sty}{\usepackage{footnotehyper}}{\usepackage{footnote}}
\makesavenoteenv{longtable}
\usepackage{graphicx}
\makeatletter
\newsavebox\pandoc@box
\newcommand*\pandocbounded[1]{% scales image to fit in text height/width
  \sbox\pandoc@box{#1}%
  \Gscale@div\@tempa{\textheight}{\dimexpr\ht\pandoc@box+\dp\pandoc@box\relax}%
  \Gscale@div\@tempb{\linewidth}{\wd\pandoc@box}%
  \ifdim\@tempb\p@<\@tempa\p@\let\@tempa\@tempb\fi% select the smaller of both
  \ifdim\@tempa\p@<\p@\scalebox{\@tempa}{\usebox\pandoc@box}%
  \else\usebox{\pandoc@box}%
  \fi%
}
% Set default figure placement to htbp
\def\fps@figure{htbp}
\makeatother


% definitions for citeproc citations
\NewDocumentCommand\citeproctext{}{}
\NewDocumentCommand\citeproc{mm}{%
  \begingroup\def\citeproctext{#2}\cite{#1}\endgroup}
\makeatletter
 % allow citations to break across lines
 \let\@cite@ofmt\@firstofone
 % avoid brackets around text for \cite:
 \def\@biblabel#1{}
 \def\@cite#1#2{{#1\if@tempswa , #2\fi}}
\makeatother
\newlength{\cslhangindent}
\setlength{\cslhangindent}{1.5em}
\newlength{\csllabelwidth}
\setlength{\csllabelwidth}{3em}
\newenvironment{CSLReferences}[2] % #1 hanging-indent, #2 entry-spacing
 {\begin{list}{}{%
  \setlength{\itemindent}{0pt}
  \setlength{\leftmargin}{0pt}
  \setlength{\parsep}{0pt}
  % turn on hanging indent if param 1 is 1
  \ifodd #1
   \setlength{\leftmargin}{\cslhangindent}
   \setlength{\itemindent}{-1\cslhangindent}
  \fi
  % set entry spacing
  \setlength{\itemsep}{#2\baselineskip}}}
 {\end{list}}
\usepackage{calc}
\newcommand{\CSLBlock}[1]{\hfill\break\parbox[t]{\linewidth}{\strut\ignorespaces#1\strut}}
\newcommand{\CSLLeftMargin}[1]{\parbox[t]{\csllabelwidth}{\strut#1\strut}}
\newcommand{\CSLRightInline}[1]{\parbox[t]{\linewidth - \csllabelwidth}{\strut#1\strut}}
\newcommand{\CSLIndent}[1]{\hspace{\cslhangindent}#1}



\setlength{\emergencystretch}{3em} % prevent overfull lines

\providecommand{\tightlist}{%
  \setlength{\itemsep}{0pt}\setlength{\parskip}{0pt}}



 


\usepackage{float}
\usepackage{tabularray}
\usepackage[normalem]{ulem}
\usepackage{graphicx}
\usepackage{rotating}
\UseTblrLibrary{booktabs}
\UseTblrLibrary{siunitx}
\NewTableCommand{\tinytableDefineColor}[3]{\definecolor{#1}{#2}{#3}}
\newcommand{\tinytableTabularrayUnderline}[1]{\underline{#1}}
\newcommand{\tinytableTabularrayStrikeout}[1]{\sout{#1}}
\usepackage{multicol}
\usepackage{enumitem}
\usepackage{ragged2e}
\usepackage{float}
\KOMAoption{captions}{tablesignature}
\makeatletter
\@ifpackageloaded{caption}{}{\usepackage{caption}}
\AtBeginDocument{%
\ifdefined\contentsname
  \renewcommand*\contentsname{Table of contents}
\else
  \newcommand\contentsname{Table of contents}
\fi
\ifdefined\listfigurename
  \renewcommand*\listfigurename{List of Figures}
\else
  \newcommand\listfigurename{List of Figures}
\fi
\ifdefined\listtablename
  \renewcommand*\listtablename{List of Tables}
\else
  \newcommand\listtablename{List of Tables}
\fi
\ifdefined\figurename
  \renewcommand*\figurename{Figure}
\else
  \newcommand\figurename{Figure}
\fi
\ifdefined\tablename
  \renewcommand*\tablename{Table}
\else
  \newcommand\tablename{Table}
\fi
}
\@ifpackageloaded{float}{}{\usepackage{float}}
\floatstyle{ruled}
\@ifundefined{c@chapter}{\newfloat{codelisting}{h}{lop}}{\newfloat{codelisting}{h}{lop}[chapter]}
\floatname{codelisting}{Listing}
\newcommand*\listoflistings{\listof{codelisting}{List of Listings}}
\makeatother
\makeatletter
\makeatother
\makeatletter
\@ifpackageloaded{caption}{}{\usepackage{caption}}
\@ifpackageloaded{subcaption}{}{\usepackage{subcaption}}
\makeatother
\makeatletter
\@ifpackageloaded{tcolorbox}{}{\usepackage[skins,breakable]{tcolorbox}}
\makeatother
\makeatletter
\@ifundefined{shadecolor}{\definecolor{shadecolor}{rgb}{.97, .97, .97}}{}
\makeatother
\makeatletter
\@ifundefined{codebgcolor}{\definecolor{codebgcolor}{HTML}{e3e3e3}}{}
\makeatother
\makeatletter
\ifdefined\Shaded\renewenvironment{Shaded}{\begin{tcolorbox}[boxrule=0pt, colback={codebgcolor}, enhanced, sharp corners, frame hidden, breakable]}{\end{tcolorbox}}\fi
\makeatother
\usepackage{bookmark}
\IfFileExists{xurl.sty}{\usepackage{xurl}}{} % add URL line breaks if available
\urlstyle{same}
\hypersetup{
  pdftitle={Problem Set Answer Template},
  pdfauthor={Your Name},
  colorlinks=true,
  linkcolor={blue},
  filecolor={Maroon},
  citecolor={Blue},
  urlcolor={Blue},
  pdfcreator={LaTeX via pandoc}}


\title{Problem Set Answer Template}
\author{Your Name}
\date{July 17, 2025}
\begin{document}
\maketitle


% these are some useful custom latex operators you can use.

\newcommand{\E}{\mathbb{E}}
\newcommand{\Prob}{\textrm{Pr}}
\newcommand{\var}{\mathbb{V}}
\newcommand{\cov}{\mathrm{cov}}
\newcommand{\corr}{\mathrm{corr}}
\newcommand{\argmin}{\arg\!\min}
\newcommand{\argmax}{\arg\!\max}
\newcommand{\qedknitr}{\hfill\rule{1.2ex}{1.2ex}}
\newcommand{\given}{\:\vert\:}
\newcommand{\indep}{\mbox{$\perp\!\!\!\perp$}}

\subsection{Question 1}\label{question-1}

\emph{(placehoder for analytical question)}

\textbf{Answer:}

We need to prove the additive property of expectations: if \(X\) and
\(Y\) are random variables, then \(\E[X + Y] = \E[X] + \E[Y]\).

By the definition of expectation, we have:

\begin{itemize}
\item
  The expectation of the sum \(X + Y\) is given by:

  \begin{equation}\phantomsection\label{eq-expect-sum}{
   \E[X + Y] = \int_{-\infty}^\infty (x + y) f_{X,Y}(x, y) \, dx \, dy,
   }\end{equation}
\item
  Using the linearity of integration, we can separate the terms inside
  the integral in Equation~\ref{eq-expect-sum}:

  \[
   \E[X + Y] = \int_{-\infty}^\infty x f_X(x) \, dx + \int_{-\infty}^\infty y f_Y(y) \, dy.
   \]
\item
  Simplifying, we obtain:

  \[
   \E[X + Y] = \E[X] + \E[Y]. \hspace{20pt} \qedknitr
   \]
\end{itemize}

\subsection{Question 2}\label{question-2}

\emph{(placeholder for question involving data analyses in {R})}

\textbf{Answer:}

We use Angrist and Pischke (\citeproc{ref-angrist2009mostly}{2009}) in
the answer. Table~\ref{tbl-variables} describes the variables we will
simulate.

\begin{longtable}[]{@{}
  >{\centering\arraybackslash}p{(\linewidth - 2\tabcolsep) * \real{0.1000}}
  >{\raggedright\arraybackslash}p{(\linewidth - 2\tabcolsep) * \real{0.8000}}@{}}
\toprule\noalign{}
\begin{minipage}[b]{\linewidth}\centering
Variable
\end{minipage} & \begin{minipage}[b]{\linewidth}\raggedright
Description
\end{minipage} \\
\midrule\noalign{}
\endfirsthead
\toprule\noalign{}
\begin{minipage}[b]{\linewidth}\centering
Variable
\end{minipage} & \begin{minipage}[b]{\linewidth}\raggedright
Description
\end{minipage} \\
\midrule\noalign{}
\endhead
\bottomrule\noalign{}
\tabularnewline
\caption{Variable description.}\label{tbl-variables}\tabularnewline
\endlastfoot
\(X_1\) & Random variable drawn from a normal distribution with mean
\(5\) and standard deviation \(2\). \\
\(X_2\) & Categorical variable with levels \texttt{"A"}, \texttt{"B"},
and \texttt{"C"}, sampled randomly with replacement. \\
\(X_3\) & Binary variable drawn from a Bernoulli distribution with
probability \(0.5\). \\
\(Y\) & Dependent variable calculated as
\texttt{3\ +\ 2\ *\ x1\ +\ 1.5\ *\ as.numeric(x2)\ +\ 0.8\ *\ x3} plus
some random noise. \\
\end{longtable}

Code below simulates a dataset with 100 observations according to the
description in Table~\ref{tbl-variables} and adds some random
missingness to variables \(X_2\) and \(X_3\). Table~\ref{tbl-gen-data}
shows the top 6 rows of this dataset.

\small

\begin{Shaded}
\begin{Highlighting}[numbers=left,,]
\CommentTok{\# Generate a sample dataset}
\NormalTok{data }\OtherTok{\textless{}{-}}\NormalTok{ dplyr}\SpecialCharTok{::}\FunctionTok{tibble}\NormalTok{(}
   \AttributeTok{x1 =} \FunctionTok{rnorm}\NormalTok{(}\DecValTok{100}\NormalTok{, }\AttributeTok{mean =} \DecValTok{5}\NormalTok{, }\AttributeTok{sd =} \DecValTok{2}\NormalTok{),}
   \AttributeTok{x2 =} \FunctionTok{factor}\NormalTok{(}\FunctionTok{sample}\NormalTok{(}\FunctionTok{c}\NormalTok{(}\StringTok{"A"}\NormalTok{, }\StringTok{"B"}\NormalTok{, }\StringTok{"C"}\NormalTok{), }\DecValTok{100}\NormalTok{, }\AttributeTok{replace =} \ConstantTok{TRUE}\NormalTok{)),}
   \AttributeTok{x3 =} \FunctionTok{rbinom}\NormalTok{(}\DecValTok{100}\NormalTok{, }\DecValTok{1}\NormalTok{, }\AttributeTok{prob =} \FloatTok{0.5}\NormalTok{),}
   \AttributeTok{y =} \DecValTok{3} \SpecialCharTok{+} \DecValTok{2} \SpecialCharTok{*}\NormalTok{ x1 }\SpecialCharTok{+} \FloatTok{1.5} \SpecialCharTok{*} \FunctionTok{as.numeric}\NormalTok{(x2) }\SpecialCharTok{+} \FloatTok{0.8} \SpecialCharTok{*}\NormalTok{ x3 }\SpecialCharTok{+} \FunctionTok{rnorm}\NormalTok{(}\DecValTok{100}\NormalTok{)}
\NormalTok{)}

\CommentTok{\# Introduce some NAs into x2 and x3}
\NormalTok{data}\SpecialCharTok{$}\NormalTok{x2[}\FunctionTok{sample}\NormalTok{(}\DecValTok{1}\SpecialCharTok{:}\DecValTok{100}\NormalTok{, }\DecValTok{10}\NormalTok{)] }\OtherTok{\textless{}{-}} \ConstantTok{NA}
\NormalTok{data}\SpecialCharTok{$}\NormalTok{x3[}\FunctionTok{sample}\NormalTok{(}\DecValTok{1}\SpecialCharTok{:}\DecValTok{100}\NormalTok{, }\DecValTok{10}\NormalTok{)] }\OtherTok{\textless{}{-}} \ConstantTok{NA}

\NormalTok{dplyr}\SpecialCharTok{::}\FunctionTok{sample\_n}\NormalTok{(data, }\AttributeTok{size =} \DecValTok{6}\NormalTok{) }\SpecialCharTok{|\textgreater{}}
\NormalTok{   knitr}\SpecialCharTok{::}\FunctionTok{kable}\NormalTok{(}
      \AttributeTok{format =} \StringTok{"latex"}\NormalTok{, }\AttributeTok{digits =} \DecValTok{3}\NormalTok{,}
      \AttributeTok{align =} \StringTok{"c"}\NormalTok{, }\AttributeTok{booktabs =} \ConstantTok{TRUE}\NormalTok{, }\AttributeTok{linesep =} \StringTok{""}
\NormalTok{   )}
\end{Highlighting}
\end{Shaded}

\begin{table}[!h]

\centering{

\begin{tabular}{cccc}
\toprule
x1 & x2 & x3 & y\\
\midrule
3.851 & C & 0 & 15.340\\
4.119 & A & 1 & 13.455\\
9.832 & C & 0 & 26.608\\
9.141 & A & NA & 20.527\\
4.049 & C & 1 & 15.979\\
3.553 & A & 1 & 14.349\\
\bottomrule
\end{tabular}

}

\caption{\label{tbl-gen-data}Simulated data.}

\end{table}%

\normalsize

\small

\normalsize

Code below produces Table~\ref{tbl-regression} that shows point and
uncertainty estimates for the model
\(Y = \alpha_1 + \beta_1 X_1 + \varepsilon\) (column 1) and
\(Y = \alpha_2 + \beta_2 X_1 + \beta_3 X_2 + \beta_4 X_3 + \varepsilon\)
(column 2).

\small

\begin{Shaded}
\begin{Highlighting}[numbers=left,,]
\NormalTok{modelsummary}\SpecialCharTok{::}\FunctionTok{modelsummary}\NormalTok{(}
   \FunctionTok{list}\NormalTok{(mod1, mod2), }\AttributeTok{stars =} \ConstantTok{TRUE}\NormalTok{,}
   \AttributeTok{gof\_omit =} \StringTok{"BIC|AIC|RMSE"}\NormalTok{,}
   \AttributeTok{coef\_omit =} \StringTok{"(Intercept)"}\NormalTok{, }\AttributeTok{output =} \StringTok{"latex"}\NormalTok{)}
\end{Highlighting}
\end{Shaded}

\begin{table}[!h]

\centering{

\centering
\begin{talltblr}[         %% tabularray outer open
entry=none,label=none,
note{}={+ p \num{< 0.1}, * p \num{< 0.05}, ** p \num{< 0.01}, *** p \num{< 0.001}},
]                     %% tabularray outer close
{                     %% tabularray inner open
colspec={Q[]Q[]Q[]},
column{2,3}={}{halign=c,},
column{1}={}{halign=l,},
hline{14}={1,2,3}{solid, black, 0.05em},
}                     %% tabularray inner close
\toprule
& (1) & (2) \\ \midrule %% TinyTableHeader
x1 & \num{1.995}*** & \num{1.950}*** \\
& (\num{0.093}) & (\num{0.057}) \\
x2B &  & \num{1.201}* \\
&  & (\num{0.493}) \\
x2C &  & \num{2.967}*** \\
&  & (\num{0.391}) \\
x3 &  & \num{1.272}** \\
&  & (\num{0.395}) \\
x2B × x3 &  & \num{-0.330} \\
&  & (\num{0.643}) \\
x2C × x3 &  & \num{-1.058}+ \\
&  & (\num{0.558}) \\
Num.Obs. & \num{100} & \num{82} \\
R2 & \num{0.878} & \num{0.949} \\
R2 Adj. & \num{0.876} & \num{0.945} \\
\bottomrule
\end{talltblr}

}

\caption{\label{tbl-regression}Regression output.}

\end{table}%

\normalsize

The code below produces Figure~\ref{fig-regression} with two panels.
Figure~\ref{fig-regression-1} shows scatter plot of \(X_1\) against
\(Y\) with linear trend, while Figure~\ref{fig-regression-2} shows
histogram of categorical variable \(X_2\).

\small

\begin{Shaded}
\begin{Highlighting}[numbers=left,,]
\FunctionTok{ggplot}\NormalTok{(data, }\FunctionTok{aes}\NormalTok{(}\AttributeTok{x =}\NormalTok{ x1, }\AttributeTok{y =}\NormalTok{ y)) }\SpecialCharTok{+}
  \FunctionTok{geom\_point}\NormalTok{() }\SpecialCharTok{+}
  \FunctionTok{geom\_smooth}\NormalTok{(}\AttributeTok{method =} \StringTok{"lm"}\NormalTok{, }\AttributeTok{se =} \ConstantTok{FALSE}\NormalTok{) }\SpecialCharTok{+}
  \FunctionTok{theme\_minimal}\NormalTok{(}\AttributeTok{base\_size =} \DecValTok{10}\NormalTok{)}

\FunctionTok{ggplot}\NormalTok{(data, }\FunctionTok{aes}\NormalTok{(}\AttributeTok{x =}\NormalTok{ x2)) }\SpecialCharTok{+}
   \FunctionTok{geom\_bar}\NormalTok{() }\SpecialCharTok{+}
   \FunctionTok{theme\_minimal}\NormalTok{(}\AttributeTok{base\_size =} \DecValTok{10}\NormalTok{) }\SpecialCharTok{+}
   \FunctionTok{labs}\NormalTok{(}\AttributeTok{x =} \StringTok{"x2 Categories"}\NormalTok{, }\AttributeTok{y =} \StringTok{"Count"}\NormalTok{, }\AttributeTok{title =} \StringTok{""}\NormalTok{)}
\end{Highlighting}
\end{Shaded}

\begin{figure}[H]

\begin{minipage}{\linewidth}

\centering{

\pandocbounded{\includegraphics[keepaspectratio]{ps_template_files/figure-pdf/fig-regression-1.pdf}}

}

\subcaption{\label{fig-regression-1}Scatter plot of \(X_1\) vs \(Y\)
variable and linear trend.}

\end{minipage}%
\newline
\begin{minipage}{\linewidth}

\centering{

\pandocbounded{\includegraphics[keepaspectratio]{ps_template_files/figure-pdf/fig-regression-2.pdf}}

}

\subcaption{\label{fig-regression-2}Histogram of \(X_2\) variable.}

\end{minipage}%

\caption{\label{fig-regression}Example figures.}

\end{figure}%

\normalsize

\clearpage

\subsection{\texorpdfstring{References
\vspace{-1em}}{References }}\label{references}

\phantomsection\label{refs}
\begin{CSLReferences}{1}{0}
\bibitem[\citeproctext]{ref-angrist2009mostly}
Angrist, Joshua D, and Jörn-Steffen Pischke. 2009. \emph{Mostly Harmless
Wconometrics: An Wmpiricist's Companion}. Princeton University Press.

\end{CSLReferences}




\end{document}
